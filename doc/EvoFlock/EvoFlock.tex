%%%%%%%%%%%%%%%%%%%%%%%%%%%%%%%%%%%%%%%%%%%%%%%%%%%%%%%%%%%%%%%%%%%%%%%%%%%
%
% Evoflock: evolved inverse design of multi-agent motion
% speculative draft paper
%
% May 29, 2025  Begin draft.
%
%%%%%%%%%%%%%%%%%%%%%%%%%%%%%%%%%%%%%%%%%%%%%%%%%%%%%%%%%%%%%%%%%%%%%%%%%%%

\documentclass[letterpaper]{article}
\usepackage{natbib,alifeconf}

% Used for ALife, not sure I still need them.
\usepackage{calc}
\usepackage[hyphens]{xurl}
\usepackage{hyperref}
\usepackage{tabularx}

% Added 20230421 to allow SIGGRAPH-style “teaser figure'' under title.
\usepackage{authblk}
\usepackage{titlepic}
\usepackage{caption}
\usepackage{float}
\usepackage[T1]{fontenc} % ??? QQQ -- "<"

%%%%%%%%%%%%%%%%%%%%%%%%%%%%%%%%%%%%%%%%%%%%%%%%%%%%%%%%%%%%%%%%%%%%%%%%%%%

% \graphicspath{ {images/} {images/fcd5/} }
\graphicspath{ {images/} }

%% For introducing terms which have a special meaning in this work.
\newcommand{\jargon}[1]{\textit{#1}}

%% Use like: {\runID backyard\_oak\_20230113\_2254}
\newcommand{\runID}{\footnotesize}

%% for laying out a row of 4, 6, or 9 images
\newcommand{\igfour}[1]{\includegraphics[width=0.24\linewidth]{#1}}
\newcommand{\igsix}[1]{\includegraphics[width=0.16\linewidth]{#1}}
\newcommand{\ignine}[1]{\includegraphics[width=0.104\linewidth]{#1}}

% small fixed-width font
\newcommand{\stt}[1]{{\small \texttt{#1}}}

%%%%%%%%%%%%%%%%%%%%%%%%%%%%%%%%%%%%%%%%%%%%%%%%%%%%%%%%%%%%%%%%%%%%%%%%%%%

\begin{document}
\title{Evoflock: evolved inverse design of multi-agent motion}
\author{Craig Reynolds\authorcr
    unaffiliated researcher\authorcr 
    cwr@red3d.com}


%%%%%%%%%%%%%%%%%%%%%%%%%%%%%%%%%%%%%%%%%%%%%%%%%%%%%%%%%%%%%%%%%%%%%%%%%%%

\captionsetup{hypcap=false}

\titlepic{\igfour{20221121_1819_step_6464.png} \hfill \igfour{20221108_2018_step_6562.png} \hfill\igfour{20221215_step_7182.png}\hfill\igfour{20221216_step_5997.png} \captionof{figure}{...random unrelated figure...} 
\label{fig:teaser}}

% Remove today's date being inserted after the title/author information.
\date{}

%% Lay out the single column top matter defined above.
\maketitle

% This puts a page number at the bottom center, but too close to text.
% \pagestyle{plain}
% \pagenumbering{arabic}

%%%%%%%%%%%%%%%%%%%%%%%%%%%%%%%%%%%%%%%%%%%%%%%%%%%%%%%%%%%%%%%%%%%%%%%%%%%

\begin{abstract}
    This paper describes an automatic method for adjusting boid-like models. Simulating the motion of bird flocks, human crowds, vehicle traffic, and other multi-agent systems is a widely used computational technique. These simulations model the behavior of a group member (bird, human, or vehicle). The group behaviors (flock, crowd, traffic) emerge from interactions between group members. These models tend to have many numeric control parameters. While each parameter is understandable in isolation, their interaction can be complex and nonlinear. It can be difficult to know how to adjust which parameters to make a desired change in the group behavior. Changing one aspect of group behavior often causes other aspects to change, leading to a long process of incremental changes. In this work, the desired group behavior is measured with an objective(/fitness/loss) function and optimized with a genetic algorithm.
\end{abstract}

\noindent{\small\textbf{Keywords:} x, y z}


\section{Introduction}
\label{sec:intro}

...the boids model \citep{reynolds_flocks_1987}...
\par

\section{Related Work}
\label{sec:related}

xxx
\par

\section{Discussion}
\label{sec:discuss}

xxx
\par

\section{Limitations}
\label{sec:limitations}

\section{Future Work}
\label{sec:future}

xxx
\par

\section{Acknowledgements}
\label{sec:ack}

...Gilbert, Matthew, Jennifer, Wahrman(?)...
\par


\bibliographystyle{apalike}
\bibliography{EvoFlock.bib}


% Appendix / Supplemental Materials

\appendix
\onecolumn
\section{Appendix}
\label{sec:appendix}

\end{document}